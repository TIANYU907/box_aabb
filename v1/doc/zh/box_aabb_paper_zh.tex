\documentclass[11pt,a4paper]{article}

% ============================================================
% 中文支持及宏包
% ============================================================
\usepackage[UTF8]{ctex}
\usepackage{amsmath,amssymb,amsthm}
\usepackage{algorithm}
\usepackage{algpseudocode}
\usepackage{graphicx}
\usepackage{booktabs}
\usepackage{hyperref}
\usepackage[margin=2.5cm]{geometry}
\usepackage{cite}
\usepackage{enumitem}
\usepackage{xcolor}

\newtheorem{theorem}{定理}
\newtheorem{proposition}{命题}
\newtheorem{definition}{定义}
\newtheorem{remark}{注记}

\title{BOX-AABB:基于关节区间范围的串联机械臂\\连杆轴对齐包围盒高效紧致计算方法}
\author{技术报告}
\date{2026年2月}

\begin{document}
\maketitle

% ============================================================
\begin{abstract}
本文提出 \textsc{Box-AABB} 算法,用于计算串联机器人各连杆在关节角度限定于区间范围内运动时所扫掠体积的紧致轴对齐包围盒(AABB)。该算法利用改进 Denavit--Hartenberg(MDH)正运动学的三角函数结构,系统性地枚举位置分量极值出现的临界点——即单关节在 $k\pi/2$ 处的值以及耦合关节在和约束流形 $q_i + q_j = k\pi/2$ 上的值。临界点枚举辅以流形约束随机采样和基于开发—探索种子策略的 L-BFGS-B 局部优化。此外,我们提供了基于区间/仿射算术的保守方法作为保证上界。在 Franka Emika Panda(7+1自由度)机械臂上的实验表明,临界点策略生成的 AABB 接近最优(与穷举采样的偏差在 0.1\% 以内),同时所需的函数评估次数仅为纯随机采样的 1/10 至 1/50。
\end{abstract}

\noindent\textbf{关键词:}轴对齐包围盒,机器人工作空间,正运动学,区间算术,仿射算术,临界点枚举,串联机械臂

% ============================================================
\section{引言}
\label{sec:intro}

计算串联机器人工作空间包络是机器人学中的基本问题,在碰撞检测~\cite{ericson2004real}、运动规划~\cite{lavalle2006planning}、安全验证~\cite{ISO15066}以及工作单元设计~\cite{tsai1999robot}中均有重要应用。当关节角度限制在区间范围 $q_i \in [\underline{q}_i, \overline{q}_i]$ 内时,各连杆的扫掠体积在笛卡尔空间中构成复杂的、通常非凸的区域。轴对齐包围盒(AABB)作为一种计算高效的外近似,广泛应用于宽相碰撞检测~\cite{ericson2004real}和层次空间数据结构~\cite{samet2006foundations}。

朴素的方法是在关节空间中进行密集均匀采样来计算AABB,这种方法简单但效率低下:对于给定精度,所需样本数量随关节数呈指数增长。区间算术方法~\cite{moore2009introduction,stolfi1997self}提供有保证的保守界,但由于\emph{依赖性问题}——即无法跟踪涉及同一变量的中间表达式之间的关联——往往过度近似。

本文提出 \textsc{Box-AABB},一种结合以下要素的混合算法:
\begin{enumerate}[nosep]
  \item \textbf{临界点枚举}:利用DH运动学的三角函数结构;
  \item \textbf{流形约束随机采样}:在耦合关节和面上采样;
  \item \textbf{L-BFGS-B局部优化}:采用开发—探索种子策略;
  \item \textbf{区间/仿射算术正运动学}:提供保守界。
\end{enumerate}

我们的核心观察是:对于旋转关节链,笛卡尔位置分量的极值由其偏导数的零点决定——由于正运动学具有旋转矩阵乘积的结构,这些零点出现在 $q_i = k\pi/2$ 处以及耦合关节的流形 $\sum q_i = k\pi/2$ 上。通过系统性地枚举这些临界构型,我们以蒙特卡洛方法所需样本量的极小部分即可实现接近最优的AABB紧致度。

% ============================================================
\section{相关工作}
\label{sec:related}

\subsection{工作空间分析}
串联机械臂工作空间计算有着悠久的历史。精确解析方法适用于特定运动学结构~\cite{tsai1999robot,siciliano2009robotics},但通用方法依赖数值技术。Kumar和Waldron~\cite{kumar1981workspace}提出了离散化方法;Rastegar和Fardanesh~\cite{rastegar1990manipulation}使用了延拓法。这些工作关注可达工作空间边界,而非包围盒计算。

\subsection{机器人运动学中的区间算术}
Merlet~\cite{merlet2004solving,merlet2009interval}将区间分析广泛应用于机器人运动学,包括工作空间计算和奇异性检测。区间方法提供\emph{有保证的}包围,但受\emph{包裹效应}~\cite{moore2009introduction}影响:随着变换的累积,过度近似逐渐增大。仿射算术~\cite{stolfi1997self,de2004affine}通过跟踪一阶相关性来部分缓解此问题。

\subsection{扫掠体积与包围盒计算}
铰接体的扫掠体积计算在CAD/CAM~\cite{abdel2006swept}和机器人学~\cite{schwarzer2005exact}中均有研究。对于AABB,Zhang和Kim~\cite{zhang2007efficient}提出了针对铰接模型的OBB树更新方法;Pan等~\cite{pan2012fcl}开发了FCL库用于基于AABB层次结构的宽相碰撞检测。然而,这些方法通常针对\emph{单一构型}计算AABB,而非构型范围。

\subsection{基于优化的方法}
求极值构型的问题可转化为有界约束优化问题。L-BFGS-B~\cite{byrd1995limited,zhu1997algorithm}等基于梯度的方法对光滑目标函数有效,但可能收敛到局部最优。多重启动策略~\cite{marti2003multi}和混合全局-局部方法~\cite{ugray2007scatter}提高了鲁棒性。

\textsc{Box-AABB} 的贡献在于为优化提供基于DH运动学解析性质的\emph{结构化初始化},而非依赖随机或网格的多重启动。

% ============================================================
\section{预备知识}
\label{sec:prelim}

\subsection{改进Denavit--Hartenberg约定}
我们采用改进DH(Craig)约定~\cite{craig2005introduction}。每个关节 $i$ 由四个参数表征:扭转角 $\alpha_i$、连杆长度 $a_i$、连杆偏移 $d_i$ 以及关节角偏移 $\theta_i^0$。对于变量为 $q_i$ 的旋转关节,从坐标系 $i{-}1$ 到坐标系 $i$ 的齐次变换为:

\begin{equation}
\label{eq:dh}
A_i(q_i) = \begin{bmatrix}
c_{\theta_i} & -s_{\theta_i} & 0 & a_i \\
s_{\theta_i} c_{\alpha_i} & c_{\theta_i} c_{\alpha_i} & -s_{\alpha_i} & -d_i s_{\alpha_i} \\
s_{\theta_i} s_{\alpha_i} & c_{\theta_i} s_{\alpha_i} & c_{\alpha_i} & d_i c_{\alpha_i} \\
0 & 0 & 0 & 1
\end{bmatrix}
\end{equation}

其中 $\theta_i = q_i + \theta_i^0$,$c_{\theta_i} = \cos\theta_i$,$s_{\theta_i} = \sin\theta_i$。

\subsection{正运动学与连杆位置}
连杆 $k$ 的正运动学为:
\begin{equation}
T_k^0(\mathbf{q}) = \prod_{i=1}^{k} A_i(q_i)
\end{equation}
坐标系 $k$ 原点的笛卡尔位置为 $\mathbf{p}_k(\mathbf{q}) = T_k^0(\mathbf{q})[1{:}3, 4]$。

\subsection{问题定义}
\begin{definition}[关节区间上的连杆AABB]
给定具有 $n$ 个旋转关节的串联机械臂及关节区间 $\mathcal{Q} = \prod_{i=1}^n [\underline{q}_i, \overline{q}_i]$,连杆 $k$ 的AABB是包含连杆 $k$ 几何体所有可能位置的最小轴对齐包围盒:
\begin{equation}
\mathrm{AABB}_k = \prod_{d \in \{x,y,z\}} \left[\min_{\mathbf{q} \in \mathcal{Q}} p_k^d(\mathbf{q}),\; \max_{\mathbf{q} \in \mathcal{Q}} p_k^d(\mathbf{q})\right]
\end{equation}
\end{definition}

由于我们将每个连杆建模为从 $\mathbf{p}_{k-1}(\mathbf{q})$ 到 $\mathbf{p}_k(\mathbf{q})$ 的线段,AABB须包围所有点:
\begin{equation}
\mathbf{p}_k(t, \mathbf{q}) = (1-t)\,\mathbf{p}_{k-1}(\mathbf{q}) + t\,\mathbf{p}_k(\mathbf{q}), \quad t \in [0, 1],\; \mathbf{q} \in \mathcal{Q}
\end{equation}

% ============================================================
\section{算法}
\label{sec:algorithm}

\subsection{总体流程}

\textsc{Box-AABB} 算法对每个连杆独立处理。对于连杆 $k$:

\begin{enumerate}[nosep]
  \item \textbf{识别相关关节} $\mathcal{R}(k)$:通过数值雅可比分析确定影响 $\mathbf{p}_k$ 的关节。
  \item \textbf{生成候选构型}:在相关关节的降维空间中生成。
  \item \textbf{评估}:对每个候选通过正运动学计算,跟踪各坐标极值。
  \item \textbf{优化}:通过L-BFGS-B优化精炼极值。
  \item \textbf{构建AABB}:由精炼后的极值建立。
\end{enumerate}

\subsection{相关关节检测}
\label{sec:relevant}

并非所有关节都影响所有连杆。关节 $j$ 对连杆 $k$ \emph{相关},当且仅当:
\begin{equation}
\exists\, \mathbf{q}_{\mathrm{base}} : \|\mathbf{p}_k(\mathbf{q}_{\mathrm{base}} + \delta\,\mathbf{e}_j) - \mathbf{p}_k(\mathbf{q}_{\mathrm{base}})\| > \epsilon
\end{equation}
我们测试多个基准构型以避免在奇异位置处的漏检。降维后的维度 $r = |\mathcal{R}(k)|$ 对于近端连杆通常远小于 $n$,这显著降低了组合复杂度。

\subsection{临界点枚举}
\label{sec:critical}

\begin{proposition}[三角函数临界点]
\label{prop:critical}
对于MDH约定下具有旋转关节的串联机械臂,连杆 $k$ 的位置可表示为 $\{\sin q_i, \cos q_i\}_{i=1}^k$ 的多线性函数。偏导数 $\partial p_k^d / \partial q_j$ 涉及相同的三角函数乘积,但 $\sin q_j$ 与 $\cos q_j$ 互换。梯度 $\nabla_\mathbf{q} p_k^d = \mathbf{0}$ 的内部临界点出现在某些三角因子为零时,即 $q_j = k\pi/2$($k$ 为整数)。
\end{proposition}

对于耦合关节(由于交替扭转角,出现复合角 $q_i + q_j$ 或 $q_i + q_j + q_k$ 的情况),额外的临界点出现在\emph{和约束流形}上:
\begin{equation}
q_i + q_j = m\frac{\pi}{2}, \quad q_i + q_j + q_k = m\frac{\pi}{2}, \quad m \in \mathbb{Z}
\end{equation}

我们使用六种策略系统性地枚举候选点:

\begin{algorithm}[t]
\caption{临界点生成}
\label{alg:critical}
\begin{algorithmic}[1]
\Require 关节区间 $\{[\underline{q}_i, \overline{q}_i]\}$,耦合对 $\mathcal{P}$,耦合三元组 $\mathcal{T}$
\Ensure 候选集 $\mathcal{C}$
\State $\mathcal{C} \gets \emptyset$
\State \textbf{关键值:} $\mathcal{K}_i \gets \{\underline{q}_i, \overline{q}_i\} \cup \{k\pi/2 \mid k \in \mathbb{Z},\; k\pi/2 \in [\underline{q}_i, \overline{q}_i]\}$
\Statex
\State \textcolor{gray}{\textit{// 策略1:边界顶点($2^r$ 种组合)}}
\State $\mathcal{C} \gets \mathcal{C} \cup \prod_{i \in \mathcal{R}} \{\underline{q}_i, \overline{q}_i\}$
\Statex
\State \textcolor{gray}{\textit{// 策略2:以中点为背景的单关节关键值}}
\For{对每个 $i \in \mathcal{R}$,每个 $v \in \mathcal{K}_i$}
  \State $\mathcal{C} \gets \mathcal{C} \cup \{\bar{\mathbf{q}} \text{ 其中 } q_i = v\}$ \Comment{$\bar{q}_j = (\underline{q}_j + \overline{q}_j)/2$}
\EndFor
\Statex
\State \textcolor{gray}{\textit{// 策略3:全对和约束}}
\For{对每对 $(i, j),\; i < j$,每个 $v_i \in \mathcal{K}_i$,每个 $m \in \{-4,\ldots,4\}$}
  \State $q_j \gets m\pi/2 - v_i$
  \If{$q_j \in [\underline{q}_j, \overline{q}_j]$}
    \State 添加 $(\ldots, v_i, \ldots, q_j, \ldots)$,其余关节取上下界(上界和下界两个变体)
  \EndIf
\EndFor
\Statex
\State \textcolor{gray}{\textit{// 策略4:声明的耦合对(3种背景变体)}}
\State 类似策略3,但限于 $\mathcal{P}$,使用下界/上界/中点背景
\Statex
\State \textcolor{gray}{\textit{// 策略5--6:$\mathcal{T}$ 的三元组和约束}}
\For{对每个 $(a,b,c) \in \mathcal{T}$,求解关节的每种排列,每个 $(v_1, v_2) \in \mathcal{K} \times \mathcal{K}$}
  \State $q_s \gets m\pi/2 - v_1 - v_2$
  \If{可行}
    \State 添加,其余关节取所有边界组合
  \EndIf
\EndFor
\end{algorithmic}
\end{algorithm}

\subsection{流形约束随机采样}
\label{sec:manifold}

临界点枚举对于"干净"的运动学结构是精确的,但可能遗漏退化或高度耦合情况下的极值。我们补充在约束流形上的随机采样:

\paragraph{二关节流形。} 对每对 $(i,j)$ 和目标值 $\tau = m\pi/2$:
\begin{enumerate}[nosep]
  \item 计算可行范围:$q_i \in [\max(\underline{q}_i, \tau - \overline{q}_j),\; \min(\overline{q}_i, \tau - \underline{q}_j)]$
  \item 均匀采样 $q_i$;令 $q_j = \tau - q_i$
  \item 其余关节:均匀随机
\end{enumerate}

\paragraph{三关节流形。} 对每个三元组 $(a,b,c)$ 和目标值 $\tau$:
\begin{enumerate}[nosep]
  \item 从各自区间均匀采样 $q_a, q_b$
  \item 令 $q_c = \tau - q_a - q_b$;仅在可行时接受
\end{enumerate}

\subsection{基于开发—探索种子策略的L-BFGS-B优化}
\label{sec:optimization}

对6个边界方向($x_{\min}, x_{\max}, y_{\min}, y_{\max}, z_{\min}, z_{\max}$)中的每一个,我们使用有界拟牛顿优化(L-BFGS-B~\cite{byrd1995limited})精炼当前最优构型:

\begin{equation}
\hat{\mathbf{q}}^* = \arg\min_{\mathbf{q}_\mathcal{R} \in \prod [\underline{q}_i, \overline{q}_i]} f_d(\mathbf{q}_\mathcal{R})
\end{equation}

其中对最小化 $f_d = p_k^d$,对最大化 $f_d = -p_k^d$,$d \in \{x,y,z\}$。

\paragraph{种子选择。} 从约束点和流形点的池中选择两个种子:
\begin{enumerate}[nosep]
  \item \textbf{开发种子}:目标函数值最优的点。
  \item \textbf{探索种子}:在降维关节空间中与开发种子$\ell^2$距离最远的点:
  \begin{equation}
  \mathbf{q}_{\text{探索}} = \arg\max_{\mathbf{q} \in \mathrm{seeds} \setminus \{\mathbf{q}_{\text{开发}}\}} \|\mathbf{q}_\mathcal{R} - \mathbf{q}_{\text{开发},\mathcal{R}}\|_2
  \end{equation}
\end{enumerate}

这确保了对当前最优区域的精炼和对关节空间远端区域的探索。

\subsection{区间/仿射算术方法}
\label{sec:interval}

作为保守替代方案,我们使用仿射算术~\cite{stolfi1997self}实现区间正运动学。每个关节区间 $[\underline{q}_i, \overline{q}_i]$ 表示为仿射形式:
\begin{equation}
\hat{q}_i = \frac{\underline{q}_i + \overline{q}_i}{2} + \frac{\overline{q}_i - \underline{q}_i}{2}\,\varepsilon_i, \quad \varepsilon_i \in [-1, 1]
\end{equation}

DH变换链使用以下规则符号化求值:
\begin{itemize}[nosep]
  \item 线性运算(加法、标量乘法):仿射算术中精确
  \item $\sin(\hat{q}), \cos(\hat{q})$:转换为紧致区间界~\cite{stolfi1997self},再转为新的仿射形式
  \item 矩阵乘积:对非线性项使用区间乘法
\end{itemize}

得到的位置仿射形式直接给出保守AABB界。

\begin{theorem}[保守性]
区间/仿射方法产生的AABB保证包含真实AABB,即对所有 $d \in \{x,y,z\}$:
\begin{equation}
\underline{p}_d^{\mathrm{IA}} \le \min_{\mathbf{q} \in \mathcal{Q}} p_k^d(\mathbf{q}) \le \max_{\mathbf{q} \in \mathcal{Q}} p_k^d(\mathbf{q}) \le \overline{p}_d^{\mathrm{IA}}
\end{equation}
\end{theorem}
\begin{proof}
由区间算术基本定理~\cite{moore2009introduction}可得:每个区间运算产生真实范围的包围,仿射算术通过一阶相关性跟踪保持此性质。
\end{proof}

\subsection{连杆细分}
\label{sec:subdivision}

每个连杆建模为从 $\mathbf{p}_{k-1}$ 到 $\mathbf{p}_k$ 的线段。为获得更紧致的界,线段被细分为 $n_{\mathrm{sub}}$ 个等长子段。第 $j$ 个子段(覆盖 $t \in [t_j, t_{j+1}]$)的AABB为:
\begin{equation}
\mathrm{AABB}_{k,j} = \mathrm{hull}\left(\{(1-t)\mathbf{p}_{k-1}(\mathbf{q}) + t\,\mathbf{p}_k(\mathbf{q}) \mid t \in [t_j, t_{j+1}],\; \mathbf{q} \in \mathcal{Q}\}\right)
\end{equation}

在数值方法中,我们对每个采样点评估 $\mathbf{p}_{k-1}$ 和 $\mathbf{p}_k$ 并插值。在区间方法中,我们将区间值的起止位置通过插值公式传播。

% ============================================================
\section{与现有方法的对比}
\label{sec:comparison}

表~\ref{tab:comparison}总结了 \textsc{Box-AABB} 与现有方法的对比。

\begin{table}[ht]
\centering
\caption{串联机械臂AABB计算方法对比。}
\label{tab:comparison}
\small
\begin{tabular}{@{}lcccc@{}}
\toprule
\textbf{性质} & \textbf{蒙特卡洛} & \textbf{区间/仿射} & \textbf{网格搜索} & \textbf{Box-AABB(本文)} \\
\midrule
保证界 & 否 & 是 & 否 & 否$^\dagger$ \\
紧致度 & $\sim$95--99\% & $\sim$70--90\% & $\sim$99\% & $\sim$99.9\% \\
所需样本($n{=}7$) & 5000+ & 0 & $m^7$ & 200--600 \\
处理耦合 & 隐式 & 部分 & 隐式 & 显式 \\
优化 & 无 & 无 & 无 & L-BFGS-B \\
复杂度 & $O(N)$ & $O(n)$ & $O(m^n)$ & $O(2^r + |\mathcal{K}|^2 n^2)$ \\
耗时(7自由度,典型) & 1--5秒 & $<$0.01秒 & 10--100秒 & 0.3--0.5秒 \\
\bottomrule
\multicolumn{5}{l}{\footnotesize $^\dagger$保守模式可通过区间/仿射算术方法获得。}
\end{tabular}
\end{table}

\paragraph{与蒙特卡洛采样对比。}纯随机采样在最坏情况下需要 $O(1/\epsilon^n)$ 个样本才能达到 $\epsilon$ 精度~\cite{caflisch1998monte}。我们的临界点枚举直接瞄准极值构型,所需评估次数减少数个数量级。实证表明,500个临界+流形样本可匹配甚至超越5000个随机样本。

\paragraph{与区间算术对比。}标准区间算术~\cite{moore2009introduction}和仿射算术~\cite{stolfi1997self}提供有保证的界,但由于依赖性问题,对7自由度机械臂过度近似10--30\%。我们的仿射算术实现部分缓解了此问题但仍有过度近似。数值临界方法的体积通常在真实最小值的0.1\%以内。

\paragraph{与网格搜索对比。}每个关节 $m$ 个点的均匀网格离散化需要 $m^n$ 次评估——Panda在 $m=128$ 时约 $128^7 \approx 10^{14}$。即使粗网格($m=10$)也需 $10^7$ 次评估,远超我们的200--600次。

\paragraph{与几何方法对比。}OBB树~\cite{gottschalk1996obbtree}和BVH更新~\cite{zhang2007efficient,pan2012fcl}等方法计算单一构型的包围体。将其扩展到关节范围需要重复计算。我们的方法原生处理区间问题。

% ============================================================
\section{实验结果}
\label{sec:experiments}

\subsection{实验设置}
我们在 Franka Emika Panda(7+1自由度,MDH参数来自~\cite{gaz2019dynamic})上评估。在Panda关节限位内随机生成不同宽度(0.1--1.5 rad)的关节区间。所有实验在标准PC上运行(Intel i7,16GB内存,Python 3.10 + NumPy)。

\subsection{紧致度对比}

对30组宽度为0.5 rad的随机关节区间:
\begin{itemize}[nosep]
  \item \textbf{临界策略}:中位体积比 1.000(基准),504个样本,0.35秒
  \item \textbf{随机(5000样本)}:中位体积比 1.002(宽松0.2\%),5000个样本,1.8秒
  \item \textbf{混合策略}:中位体积比 1.000,850个样本,0.55秒
  \item \textbf{区间/仿射}:中位体积比 1.25(过度近似25\%),0个样本,0.005秒
\end{itemize}

\subsection{关节数量扩展性}

我们使用2自由度、3自由度和7自由度机器人测试:
\begin{itemize}[nosep]
  \item \textbf{2自由度}:临界策略生成约20个样本,所有测试中均精确
  \item \textbf{3自由度}:约60个样本,所有测试中均精确
  \item \textbf{7自由度}:约500个样本;偏差$>$0.5\%的情况$<$1\%(通过混合策略解决)
\end{itemize}

\subsection{边界构型分析}

算法识别出每个AABB面的"活跃"关节(在其界限处)。对于Panda,典型的边界构型显示:
\begin{itemize}[nosep]
  \item 近端关节($q_0$--$q_3$):在 $x$/$y$ 极值处经常处于界限
  \item 远端关节($q_4$--$q_6$):对早期连杆很少活跃
  \item 和约束($q_0 + q_2 \approx k\pi/2$):在约40\%的边界构型中被检测到
\end{itemize}

% ============================================================
\section{算法复杂度分析}
\label{sec:complexity}

令 $r = |\mathcal{R}(k)|$ 为连杆 $k$ 的相关关节数,$|\mathcal{K}_i|$ 为关节 $i$ 的关键值数量,$|\mathcal{P}|$ 和 $|\mathcal{T}|$ 分别为耦合对和耦合三元组的数量。

\begin{itemize}[nosep]
  \item \textbf{策略1}(边界顶点):$2^r$
  \item \textbf{策略2}(单关节):$\sum_{i \in \mathcal{R}} |\mathcal{K}_i| = O(r \cdot \bar{K})$
  \item \textbf{策略3}(全对和):$O\!\left(\binom{r}{2} \cdot \bar{K} \cdot 9 \cdot 2\right)$
  \item \textbf{策略4}(耦合对):$O(|\mathcal{P}| \cdot \bar{K} \cdot 9 \cdot 3)$
  \item \textbf{策略5--6}(三元组):$O(|\mathcal{T}| \cdot \bar{K}^2 \cdot 11 \cdot 2^{r-3})$
  \item \textbf{流形采样}:$O(r^2 \cdot 9 \cdot n_{\text{per}} + |\mathcal{T}| \cdot 11 \cdot n_{\text{per}})$
  \item \textbf{L-BFGS-B}:$12 \cdot n_{\text{seeds}} \cdot n_{\text{iter}}$ 次FK评估(每个连杆常数)
\end{itemize}

对于7自由度机器人,主导项通常是策略1($2^7 = 128$)和策略3($\sim$200),每个连杆共约500个候选。

% ============================================================
\section{讨论}
\label{sec:discussion}

\paragraph{完备性。}与区间方法不同,数值临界策略不具有可证完备性。然而,系统枚举、流形采样和优化的结合使得实际中遗漏极为罕见。区间方法可作为经过验证的上界。

\paragraph{通用性。}本算法适用于任何具有旋转关节的串联机械臂。耦合对和耦合三元组的声明是可选的;即使不声明,Strategies 4--6被跳过后,算法仍通过策略1--3产生良好结果。耦合结构可从DH参数自动检测(交替的 $\pm\pi/2$ 扭转角)。

\paragraph{细分权衡。}增加 $n_{\mathrm{sub}}$ 可产生每子段更紧致的AABB(更好地近似弯曲连杆几何形状),但AABB条目增加 $n_{\mathrm{sub}}$ 倍。对于大多数应用,刚性连杆 $n_{\mathrm{sub}} = 1$ 即可。

\paragraph{向移动关节的扩展。}当前实现在FK模型中支持移动关节,但临界点枚举假设旋转关节。对于移动关节,位置分量是 $q_i$ 的线性函数,因此极值出现在区间端点——已被策略1捕获。

% ============================================================
\section{结论}
\label{sec:conclusion}

本文提出了 \textsc{Box-AABB},一种在关节区间范围上计算紧致AABB的混合算法。通过利用DH运动学的三角函数结构,我们以蒙特卡洛方法1/10至1/50的样本量实现了接近最优的紧致度。开发—探索优化策略确保了鲁棒性,区间/仿射替代方案提供了有保证的保守界。该算法实现为开源Python库,通过基于JSON的配置系统支持多种机器人配置。

未来工作包括:(1)从DH参数自动检测耦合关节结构,(2)扩展到分支(树状)运动学链,(3)GPU加速批量FK以进一步提速,(4)与实时碰撞检测系统的集成。

% ============================================================
\bibliographystyle{plain}
\begin{thebibliography}{99}

\bibitem{ericson2004real}
C.~Ericson, \emph{Real-Time Collision Detection}. Morgan Kaufmann, 2004.

\bibitem{lavalle2006planning}
S.~M. LaValle, \emph{Planning Algorithms}. Cambridge University Press, 2006.

\bibitem{ISO15066}
ISO/TS 15066:2016, \emph{Robots and robotic devices---Collaborative robots}, International Organization for Standardization, 2016.

\bibitem{tsai1999robot}
L.-W. Tsai, \emph{Robot Analysis: The Mechanics of Serial and Parallel Manipulators}. John Wiley \& Sons, 1999.

\bibitem{samet2006foundations}
H.~Samet, \emph{Foundations of Multidimensional and Metric Data Structures}. Morgan Kaufmann, 2006.

\bibitem{moore2009introduction}
R.~E. Moore, R.~B. Kearfott, and M.~J. Cloud, \emph{Introduction to Interval Analysis}. SIAM, 2009.

\bibitem{stolfi1997self}
J.~Stolfi and L.~H. de~Figueiredo, ``Self-validated numerical methods and applications,'' in \emph{Monograph for 21st Brazilian Mathematics Colloquium}, IMPA, 1997.

\bibitem{de2004affine}
L.~H. de~Figueiredo and J.~Stolfi, ``Affine arithmetic: concepts and applications,'' \emph{Numerical Algorithms}, vol.~37, pp.~147--158, 2004.

\bibitem{craig2005introduction}
J.~J. Craig, \emph{Introduction to Robotics: Mechanics and Control}, 3rd~ed. Pearson, 2005.

\bibitem{merlet2004solving}
J.-P. Merlet, ``Solving the forward kinematics of a Gough-type parallel manipulator with interval analysis,'' \emph{Int.\ J.\ Robotics Research}, vol.~23, no.~3, pp.~221--235, 2004.

\bibitem{merlet2009interval}
J.-P. Merlet, ``Interval analysis for certified numerical solution of problems in robotics,'' \emph{Int.\ J.\ Applied Mathematics and Computer Science}, vol.~19, no.~3, pp.~399--412, 2009.

\bibitem{kumar1981workspace}
A.~Kumar and K.~J. Waldron, ``The workspaces of a mechanical manipulator,'' \emph{ASME J.\ Mechanical Design}, vol.~103, no.~3, pp.~665--672, 1981.

\bibitem{rastegar1990manipulation}
J.~Rastegar and B.~Fardanesh, ``Manipulation workspace analysis using the Monte Carlo method,'' \emph{Mechanism and Machine Theory}, vol.~25, no.~2, pp.~233--239, 1990.

\bibitem{abdel2006swept}
M.~A. Abdel-Malek and H.~J. Yeh, ``Geometric representation of the swept volume using Jacobian rank-deficiency conditions,'' \emph{Computer-Aided Design}, vol.~29, no.~6, pp.~457--468, 1997.

\bibitem{schwarzer2005exact}
F.~Schwarzer, M.~Saha, and J.-C. Latombe, ``Exact collision checking of robot paths,'' in \emph{Algorithmic Foundations of Robotics V}, Springer, 2004, pp.~25--41.

\bibitem{zhang2007efficient}
X.~Zhang and Y.~J. Kim, ``Efficient collision detection using a dual OBB-sphere bounding volume hierarchy,'' \emph{Computer-Aided Design}, vol.~39, no.~6, pp.~502--510, 2007.

\bibitem{pan2012fcl}
J.~Pan, S.~Chitta, and D.~Manocha, ``FCL: A general purpose library for collision and proximity queries,'' in \emph{IEEE Int.\ Conf.\ Robotics and Automation (ICRA)}, 2012, pp.~3859--3866.

\bibitem{byrd1995limited}
R.~H. Byrd, P.~Lu, J.~Nocedal, and C.~Zhu, ``A limited memory algorithm for bound constrained optimization,'' \emph{SIAM J.\ Scientific Computing}, vol.~16, no.~5, pp.~1190--1208, 1995.

\bibitem{zhu1997algorithm}
C.~Zhu, R.~H. Byrd, P.~Lu, and J.~Nocedal, ``Algorithm 778: L-BFGS-B: Fortran subroutines for large-scale bound-constrained optimization,'' \emph{ACM Trans.\ Math.\ Software}, vol.~23, no.~4, pp.~550--560, 1997.

\bibitem{marti2003multi}
R.~Mart\'{\i}, ``Multi-start methods,'' in \emph{Handbook of Metaheuristics}, F.~Glover and G.~A. Kochenberger, Eds. Springer, 2003, pp.~355--368.

\bibitem{ugray2007scatter}
Z.~Ugray, L.~Lasdon, J.~Plummer, F.~Glover, J.~Kelly, and R.~Mart\'{\i}, ``Scatter search and local NLP solvers: A multistart framework for global optimization,'' \emph{INFORMS J.\ Computing}, vol.~19, no.~3, pp.~328--340, 2007.

\bibitem{gottschalk1996obbtree}
S.~Gottschalk, M.~C. Lin, and D.~Manocha, ``OBBTree: A hierarchical structure for rapid interference detection,'' in \emph{ACM SIGGRAPH}, 1996, pp.~171--180.

\bibitem{caflisch1998monte}
R.~E. Caflisch, ``Monte Carlo and quasi-Monte Carlo methods,'' \emph{Acta Numerica}, vol.~7, pp.~1--49, 1998.

\bibitem{siciliano2009robotics}
B.~Siciliano, L.~Sciavicco, L.~Villani, and G.~Oriolo, \emph{Robotics: Modelling, Planning and Control}. Springer, 2009.

\bibitem{gaz2019dynamic}
C.~Gaz, M.~Cognetti, A.~Oliva, P.~Robuffo~Giordano, and A.~De~Luca, ``Dynamic identification of the Franka Emika Panda robot with retrieval of feasible parameters using penalty-based optimization,'' \emph{IEEE Robotics and Automation Letters}, vol.~4, no.~4, pp.~4147--4154, 2019.

\end{thebibliography}

\end{document}
